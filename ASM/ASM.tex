\documentclass[conference]{IEEEtran}
\IEEEoverridecommandlockouts
% The preceding line is only needed to identify funding in the first footnote. If that is unneeded, please comment it out.
\usepackage[ngerman]{babel}
\usepackage{cite}
\usepackage{amsmath,amssymb,amsfonts,mathtools}
\usepackage{algorithmic}
\usepackage{graphicx}
\usepackage{textcomp}
\usepackage[dvipsnames]{xcolor}
\def\BibTeX{{\rm B\kern-.05em{\sc i\kern-.025em b}\kern-.08em
    T\kern-.1667em\lower.7ex\hbox{E}\kern-.125emX}}
\usepackage{listings}
\usepackage{tabularx}
\usepackage{booktabs}
\usepackage{siunitx}
\usepackage{float}
\usepackage{caption}

\definecolor{codegreen}{rgb}{0,0.6,0}
\definecolor{codegray}{rgb}{0.5,0.5,0.5}
\definecolor{codepurple}{rgb}{0.58,0,0.82}
\definecolor{backcolour}{rgb}{0.95,0.95,0.95}
\renewcommand{\lstlistingname}{Code}% Listing -> Algorithm
\renewcommand{\lstlistlistingname}{List of \lstlistingname s}% List of Listings -> List of Algorithms
\lstdefinestyle{mystyle}{
    backgroundcolor=\color{backcolour},
    commentstyle=\color{codegreen},
    keywordstyle=\color{magenta},
    numberstyle=\tiny\color{codegray},
    stringstyle=\color{codepurple},
    basicstyle=\ttfamily\footnotesize,
    breakatwhitespace=false,
    breaklines=true,
    captionpos=b,
    keepspaces=true,
    numbers=left,
    numbersep=3pt,
    showspaces=false,
    showstringspaces=false,
    showtabs=false,
    tabsize=3
}
\lstset{style=mystyle}

\usepackage{hyperref}
\hypersetup{
    pdftex,
    pdftitle={Versuch ASM},
    pdfsubject={Versuch ASM},
    pdfauthor=AJC,
    colorlinks,
    citecolor=black,
    filecolor=black,
    linkcolor=black,
    urlcolor=black
}



\begin{document}

\title{
    \centering
    \includegraphics[width=0.5\textwidth]{../OTHR_OTHR_Logo.pdf}\\
    \textsc{Drehstrom-Asynchronmaschine mit Schleifringläufer} \\
}

\maketitle

\begin{abstract}
    In diesem Versuch soll eine Käfigläufer-Asynchronmaschine genauer
    untersucht werden. Zur Bestimmung der ESB- Parameter wurde hierzu der
    Leerlauf, als auch Kurzschlussversuch durchgeführt\dots
\end{abstract}

\section{Anlaufmoment und Kippmoment}

Bei diesem Versuch wurde das Drehmoment mit einem 40cm langen Stab und einer
Digitalen Wage gemessen.

Mit der Gleichung \ref{eq:moment} wurde das Drehmoment aus den Messungen berechnet.

\begin{equation} \label{eq:moment}
    M=G\cdot g\cdot l = \frac{G(I_1)}{1000}\si{kg}\cdot 9,81\si{m/s^2} \cdot 0,4\si{m}
\end{equation}

In den Abbildungen \ref{fig:Anlaufmoment} und \ref{fig:Kippmoment} wird das
Anlaufmoment der ASM über die Spannung graphisch dargestellt.

Die obere Hälfte der Abbildungen zeigt eine linearisierte Extrapolation des
Verlaufs der Drehmomente $M_{an}$ und $M_{\textit{Kipp}}$ über die Spannung
$U_1^2$, bis $400^2V$.

Aber in der unteren Hälfte der Abbildungen wird das Drehmoment der ASM direkt
über die Spannung $U_1$ dargestellt und mit einer quadratischen Polynomfunktion
extrapoliert.

\begin{table}[htbp]
    \centering
    \begin{tabularx}{\columnwidth}{XXXXXXX}
    \toprule
     $I_1[A]$ &  $P_1[W]$ &  $U[V]$ &  $G[g]$ &  $M_{an}[Nm]$ \\
    \midrule
            1,0 &          28 &      27,5 &        35 &        0,137340 \\
            2,0 &          92 &      55,0 &        85 &        0,333540 \\
            3,0 &         194 &      82,0 &       222 &        0,871128 \\
            4,0 &         352 &     107,0 &       426 &        1,671624 \\
            4,2 &         392 &     113,0 &       464 &        1,820736 \\
            5,0 &         470 &     134,0 &       685 &        2,687940 \\
    \bottomrule
    \end{tabularx}
    \caption{Anlaufmoment}
\end{table}


\begin{figure}[htbp]
    \centering
    \includegraphics[width=\columnwidth]{./figures/anlaufmoment.pdf}
    \caption{Anlaufmoment}
    \label{fig:Anlaufmoment}
\end{figure}

Der lineare Verlauf in Abb.\ref{fig:Anlaufmoment}, $M_{an}(U)$ [\textcolor{red}{rot}]:
\begin{gather*}
    0,1511 x - 0,07306 \\
    M_{an}(U_N=400^2V^2)=24,11\ \si{Nm}
\end{gather*}

Und die quadratische Polynomfunktion $M_{an}(U)$ [\textcolor{blue}{blau}]:
\begin{gather*}
    0,000199 x^2 -0,007889 x + 0,190977\\
    M_{an}(U_N=400V)=28,83\ \si{Nm}
\end{gather*}

\begin{table}[htbp]
    \centering
    \begin{tabularx}{\columnwidth}{XXXXXXX}
        \toprule
             $I_1[A]$ &  $P_1[W]$ &  $U[V]$ &  $G[g]$ &  $M_{an}[Nm]$ \\
        \midrule
                1,0 &          24 &        36 &        32 &        0,125568 \\
                2,0 &         102 &        73 &       218 &        0,855432 \\
                3,0 &         380 &       108 &       508 &        1,993392 \\
                4,0 &         670 &       140 &       891 &        3,496284 \\
                4,2 &         745 &       148 &      1000 &        3,924000 \\
                5,0 &        1046 &       175 &      1425 &        5,591700 \\
        \bottomrule
    \end{tabularx}
    \caption{Kippmoment}
\end{table}


\begin{figure}[htbp]
    \centering
    \includegraphics[width=\columnwidth]{./figures/kippmoment.pdf}
    \caption{Kippmoment}
    \label{fig:Kippmoment}
\end{figure}

Der lineare Verlauf in Abb.\ref{fig:Kippmoment}, $M_{\textit{Kipp}}(U)$ [\textcolor{red}{rot}]:
\begin{gather*}
    0,1863 x - 0,1431\\
    M_{Kipp}(U_N=400^2V^2)=29,67 \ \si{Nm}
\end{gather*}

Und die quadratische Polynomfunktion $M_{\textit{Kipp}}(U)$ [\textcolor{blue}{blau}]:
\begin{gather*}
    0,000198 x^2 - 0,0024495 x -0,035842\\
    M_{Kipp}(U_N=400V)=30,6\ \si{Nm}
\end{gather*}



\subsection{Zusammenhang zwischen Drehmoment und Spannung}

Bei zunehmender Spannung $U_1$ steigt das Drehmoment M quadratisch an. Das
Drehmoment ist direkt proportional zu $U^2$ ($M \sim U^2$) und darauf ergibt sich
folgende Beziehnung: \[ M = \textit{konst.} \cdot U \]


\section{Bestimmung der Ersatzschaltbildparameter}
\subsection{Parameterbestimmung aufgrund der Kurzschlussmessung}

Umrechnen der gemessenen Widerstandswerte auf die Bezugstemperatur $T=75^\circ$C :
\begin{align*}
    R_{1_{75}} & = R_{1_{20}}\frac{235^\circ\si{C}+75^\circ\text{C}}{235^\circ\text{C}+20^\circ\text{C}}         \\
               & = 2,32\ \Omega\cdot \frac{235^\circ\si{C}+75^\circ\text{C}}{235^\circ\text{C}+20^\circ\text{C}} \\
               & = 2,820\ \Omega
\end{align*}

Berechnen des Läuferwiderstands mithilfe der Übersetzungsverhältnis:
\begin{align*}
    R_2^\prime        & = \ddot{u}^2\cdot R_2 = 4,7^2 \cdot 216\si{m\Omega}   = 4,77 \Omega                             \\
    R_{2_{75}}^\prime & = R_{2_{20}}\frac{235^\circ\si{C}+75^\circ\si{C}}{235^\circ\si{C}+20^\circ\si{C}}         \\
                      & = 4,77\ \Omega\cdot \frac{235^\circ\si{C}+75^\circ\si{C}}{235^\circ\si{C}+20^\circ\si{C}} \\
                      & = 5,8\ \Omega
\end{align*}

Es gilt:
\begin{equation} \label{eq:Kurzschlusswiderstand_gemessen}
    \boxed{R_k = R_1 + R_2^\prime = 2,820\ \Omega + 5,8\ \Omega = 8,62\ \Omega}
\end{equation}

Aus der Kurzschlussmessung in 4.2.1 wird der Kurzschlusswiderstand $R_k$ mit:

\begin{equation} \label{eq:Kurzschlusswiderstand}
    \boxed{R_k = \frac{U_k}{I_k}\cdot cos\varphi_k}
\end{equation}

\begin{table}[htbp]
    \begin{tabularx}{\columnwidth}{XXXXX}
        \toprule
        $I_1[A]$ & $P_1[W]$ & $U[V]$ & $G[g]$ & $M_{an}[Nm]$ \\
        \midrule
        4,2      & 392      & 113    & 464    & 1,82074      \\
        \bottomrule
    \end{tabularx}
    \caption{Kurzschlussmessung 4.2.1}
    \label{tab:Kurzschlussmessung}
\end{table}

Mit den Werten aus der Tabelle \ref{tab:Kurzschlussmessung} kann $cos\varphi_k$
berechnet werden:

\begin{equation}
    \boxed{cos\varphi_k = \frac{P_1}{\sqrt{3} \cdot I_1 \cdot U}}
\end{equation}

\begin{equation} \label{eq:cosphi_solved}
    cos\varphi_k = \frac{392 \si{W}}{\sqrt{3} \cdot 4,2 \si{A} \cdot 113 \si{V}} = 0,4769
\end{equation}

Da die Maschine im Stern geschaltet ist, ergibt sich für die Spannung $U_k$:

\begin{equation} \label{eq:Uk_solved}
    \boxed{U_k = \frac{U}{\sqrt{3}}} = \frac{113 \si{V}}{\sqrt{3}} = 65,24 \si{V}
\end{equation}

In der Formel \ref{eq:Kurzschlusswiderstand} einsetzen:

\begin{equation} \label{eq:Kurzschlusswiderstand_calc}
    \boxed{R_k = \frac{65,24\si{V}}{4,2\si{A}}\cdot cos(61,51^\circ) = \underline{7,409 \Omega}}
\end{equation}

Die Streureaktanzen $X_{1\sigma}$ und $X_{2\sigma}^\prime$ werden mit:

\begin{equation} \label{eq:Streureaktanzen}
    X_{1\sigma} + X_{2\sigma}^\prime = \frac{U_k}{I_k}\cdot sin\varphi_k
\end{equation}

Da $X_{1\sigma} = X_{2\sigma}^\prime$ ist, ergibt sich:

\begin{align} \label{eq:Streureaktanzen_solved}
    \Aboxed{X_{1\sigma} & = X_{2\sigma}^\prime = \frac{U_k}{I_k}\cdot\frac{1}{2}\cdot sin\varphi_k} \\
                        & = \frac{65,24\si{V}}{3,69\si{A}}\cdot\frac{1}{2}\cdot sin(61,51^\circ)    \\
                        & =6,826\ \Omega
\end{align}

\subsubsection{Vergleich zwischen ermittelter und gemessener Kurzschlusswiderstand}

Die Werte des berechnete Kurzschlusswiderstands
(\ref{eq:Kurzschlusswiderstand_gemessen}) haben eine Abweichung von ca. 15 \% zu
dem gemessenen Widerstand (\ref{eq:Kurzschlusswiderstand_calc}). Diese Differenz
ist zu einem großen Teil auf die gewählte Referenztemperatur von 75°C zurück
zuführen. Bei einer zügig durchgeführten Messung ist die Wärmeentwicklung der
Maschine geringer. Nimmt man also einen geringeren Temperaturanstieg wie etwa
60°C an kommt man dem gemessenen Wert deutlich Näher.

\begin{equation}
    R_{2_{60}}^\prime = R_{2_{20}}^\prime \cdot \dfrac{60^\circ \si{C} + 235^\circ \si{C}}{20^\circ \si{C} + 235^\circ \si{C}}
\end{equation}

\begin{equation}
    R_{2_{60}}^\prime = 4,77 \Omega \cdot \dfrac{60^\circ \si{C} + 235^\circ \si{C}}{20^\circ \si{C} + 235^\circ \si{C}} = 4,77 \Omega
\end{equation}

aus (\ref{eq:Kurzschlusswiderstand_gemessen}):
\begin{equation}
    \boxed{R_{k} = R_{1} + R_{2_{60}}^\prime = 2,82 \Omega + 4,77 \Omega = 7,59 \Omega}
\end{equation}


\subsection{Parameterbestimmung aufgrund der Leerlaufmessung}
Bei dem folgenden Versuch ist Läuferseitig ein Leerlauf und somit kein
Läuferstrom $I_{2}$ zu messen. Der fließende Strom besteht ausschließlich aus
dem Ständerstrom $I_{1}$ bzw. dem Magnetisierungsstrom $I_{0}$.
Der verwendete Hebelarm aus den letzten Messungen wird entfernt, der Läufer wird
mit kurzen Leitungen kurzgeschlossen.  Die verwendeten Werte werden Tabelle
\ref{tab:trennung_eisen_reib} entnommen.

Die gemessene Spannung wird auf die Strangspannung umgerechnet. Anschließende
wird über eine Masche der Spannungabfall $U_{h, Fe}$ berechnet.

\begin{equation}
    U_{Strang} = \dfrac{U}{ \sqrt{3} } = 230,94\si{V}
\end{equation}

\begin{align*}
    U_{h, Fe} & = U_{Strang} - U_{1} = U_{Strang} - R_{1} \cdot I_{0}        \\
              & = 230,94\si{V} - 2,32 \Omega \cdot 2,75\si{A} = 224,56\si{V}
\end{align*}


\begin{equation}
    \varphi_{0} = arcos\left( \dfrac{P_{0} - P_{Cu1} }{3 \cdot U_{Strang} \cdot I_{0} }\right) = 84,35^\circ
\end{equation}


\begin{align*}
    I_{Fe+Reib} & = I_{0} \cdot cos( \varphi ) = 0,27\si{A}   \\
    I_{m}       & = I_{0} \cdot sin( \varphi ) = 2,7367\si{A}
    %    Z_{h, Fe} = \dfrac{ U_{h, Fe} }{ I_{0} } = ( \frac{1}{X_{h}} + \frac{1}{R_{Fe}} )^{-1} = 81,66 \Omega
\end{align*}

\begin{equation}
    R_{Fe+Reib} = \dfrac{ U_{h, Fe} }{ I_{Fe+Reib} } = 831,7 \Omega
    %    \boxed{ R = Z_{h, Fe} \cdot ( \frac{1}{cos( \varphi) } +1 ) = 729,606 \Omega }
\end{equation}

\begin{equation}
    X_{h} = \dfrac{ U_{h, Fe} }{ I_{m} } = 82,06 \Omega
    %\boxed{ X_{h} = 91,95 \Omega }
\end{equation}

\subsubsection{Verhältniss aus Haupt- und Streureaktanz}

mit $X_{ \sigma } =  X_{1, \sigma } + X_{2, \sigma }^\prime$

\begin{equation}
    \dfrac{ X_{h}}{X_{ \sigma }} = \dfrac{ 82,06 \Omega }{ 13,652 \Omega } = 6,01
    %\dfrac{ X_{h}}{X_{ \sigma }} = \dfrac{ 91,95 \Omega }{ 13,652 \Omega } = 6,735
\end{equation}

Die Hauptreaktanz ist ca. 6 mal größer als die Streureaktanz.

\subsubsection{Vollständige Ersatzschaltbild [ESB]}

\begin{figure}[htbp]
    \centering
    \includegraphics[width=\columnwidth]{./figures/Vollstaendiges_ESB.jpg}
    \caption{Vollständige Ersatzschaltbild}
    \label{fig:ESB_Vollstaendig}
\end{figure}

\newpage
\section{Trennung von Eisen- und Reibungsverlusten}

\begin{table}[htbp]
    \centering
    \begin{tabularx}{\columnwidth}{XXXXXX}
        \toprule
        $I_0[A]$ &  $P_0[W]$ &  $P_{Cu}[W]$ &  $P_{Fe+Reib}$ &  $U[V]$ &  $n[min]$ \\
        & & & $[W]$& & \\
        \midrule
               2,75 &         240 &        19,1400 &            220,8600 &       400 &        1493 \\
               1,70 &         140 &        11,8320 &            128,1680 &       300 &        1490 \\
               1,10 &          80 &         7,6560 &             72,3440 &       200 &        1485 \\
               0,58 &          44 &         4,0368 &             39,9632 &       100 &        1461 \\
        \bottomrule
    \end{tabularx}
    \caption{Trennung von Eisen und Reibungsverlusten}
    \label{tab:trennung_eisen_reib}
\end{table}


\begin{figure}[h]
    \centering
    \includegraphics[width=\columnwidth]{./figures/trennung_eisen_reib.pdf}
    \caption{Trennung von Eisen- und Reibungsverlusten}
    \label{fig:trennung_von_eisen_reib}
\end{figure}

\section{$M/n$ Kennlinie}

aus (\ref{eq:Streureaktanzen_solved}), (\ref{eq:R_2^prime}) und (\ref{eq:R1_75})

\smallskip
Mit $R_1 = 2,82\Omega$:
\begin{equation} \label{eq:s_k-mit-R_1}
    s_k = \frac{R_2^{\prime}}{\sqrt{R_{1}^{2} + X_{\sigma}^{2}}} = \frac{5,80\Omega}{\sqrt{2,82^2+13,652^2}\Omega} = 0,416
\end{equation}

\begin{align}
    M_{k} & = \frac{U_{1}^{2}}{4 \pi n_{1} \left(R_{1} + \sqrt{R_{1}^{2} + X_{\sigma}^{2}}\right)}           \\
          & = \frac{400\si{V}^{2}}{4 \pi \cdot 25 \si{sec^{-1}} \cdot (2,82 + \sqrt{2,82^2+13,652^2})\Omega} \\
          & = 31,653 \si{Nm}
    \label{eq:M_k-mit-R_1}
\end{align}

\smallskip
Mit $R_1 = 0\Omega$:
\begin{equation} \label{eq:s_k-naehrung}
    s_k \approx \frac{R_2^{\prime}}{X_{\sigma}} = \frac{5,80\Omega}{13,652\Omega} = 0,425
\end{equation}

\begin{align}
    M_{k} & \approx \frac{U_{1}^{2}}{4 \pi X_{\sigma} n_{1}} \approx \frac{400\si{V}^{2}}{4 \pi\cdot 13,652 \Omega \cdot 25 \si{sec^{-1}}} \\
          & \approx 37,306 \si{Nm}
    \label{eq:M_k-mit-R_1}
\end{align}

\begin{table}[htbp]
    \centering
    \begin{tabularx}{\columnwidth}{XXX}
        \toprule
        $s$   & $M[Nm]$ & $n_1[min^{-1}]$ \\
        \midrule
        -0,50 & -29,58  & 2250            \\
        -0,25 & -30,09  & 1875            \\
        0,00  & 0       & 1500            \\
        0,25  & 30,09   & 1125            \\
        0,50  & 29,58   & 750             \\
        0,75  & 24,01   & 375             \\
        1,00  & 19,49   & 0               \\
        \bottomrule
    \end{tabularx}
    \caption{Drehmoment über die Drehzahl}
    \label{tab:sigma_vs_M_n1}
\end{table}

\begin{table}[htbp]
    \centering
    \begin{tabularx}{\columnwidth}{XXXXXXX}
        \toprule
        $s$   & $M_{kipp}[Nm]$ & $n_1[min^{-1}]$ \\
        \midrule
        -0,50 & -25,32  & 2250            \\
        -0,25 & -14,90  & 1875            \\
        0,00  & 0       & 1500            \\
        0,25  & 14,90   & 1125            \\
        0,50  & 25,32   & 750             \\
        0,75  & 30,39   & 375             \\
        1,00  & 31,65   & 0               \\
        \bottomrule
    \end{tabularx}
    \caption{Kippmoment über die Drehzahl}
    \label{tab:M_kipp-mit-laeufervorwiderstand}
\end{table}


\begin{figure}[htbp]
    \centering
    \includegraphics[width=\columnwidth]{./figures/m_n-kennlinie.pdf}
    \caption{$M/n$ Kennlinie}
    \label{fig:trennung_von_eisen_reib}
\end{figure}

\end{document}
