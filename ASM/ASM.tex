\documentclass[conference]{IEEEtran}
\IEEEoverridecommandlockouts
% The preceding line is only needed to identify funding in the first footnote. If that is unneeded, please comment it out.
\usepackage[ngerman]{babel}
\usepackage{cite}
\usepackage{amsmath,amssymb,amsfonts,mathtools}
\usepackage{algorithmic}
\usepackage{graphicx}
\usepackage{textcomp}
\usepackage[dvipsnames]{xcolor}
\def\BibTeX{{\rm B\kern-.05em{\sc i\kern-.025em b}\kern-.08em
    T\kern-.1667em\lower.7ex\hbox{E}\kern-.125emX}}
\usepackage{listings}
\usepackage{tabularx}
\usepackage{booktabs}
\usepackage{siunitx}
\usepackage{float}
\usepackage{caption}

\definecolor{codegreen}{rgb}{0,0.6,0}
\definecolor{codegray}{rgb}{0.5,0.5,0.5}
\definecolor{codepurple}{rgb}{0.58,0,0.82}
\definecolor{backcolour}{rgb}{0.95,0.95,0.95}
\renewcommand{\lstlistingname}{Code}% Listing -> Algorithm
\renewcommand{\lstlistlistingname}{List of \lstlistingname s}% List of Listings -> List of Algorithms
\lstdefinestyle{mystyle}{
    backgroundcolor=\color{backcolour},
    commentstyle=\color{codegreen},
    keywordstyle=\color{magenta},
    numberstyle=\tiny\color{codegray},
    stringstyle=\color{codepurple},
    basicstyle=\ttfamily\footnotesize,
    breakatwhitespace=false,
    breaklines=true,
    captionpos=b,
    keepspaces=true,
    numbers=left,
    numbersep=3pt,
    showspaces=false,
    showstringspaces=false,
    showtabs=false,
    tabsize=3
}
\lstset{style=mystyle}

\usepackage{hyperref}
\hypersetup{
    pdftex,
    pdftitle={Versuch ASM},
    pdfsubject={Versuch ASM},
    pdfauthor=AJC,
    colorlinks,
    citecolor=black,
    filecolor=black,
    linkcolor=black,
    urlcolor=black
}



\begin{document}

\title{
    \centering
    \includegraphics[width=0.5\textwidth]{../OTHR_OTHR_Logo.pdf}\\
    \textsc{Drehstrom-Asynchronmaschine mit Schleifringläufer} \\
}

\maketitle

\begin{abstract}
    In diesem Versuch soll eine Asynchronmaschine mit Schleifringläufer genauer
    untersucht werden. Die Versuchsmaschine wird mithilfe einer
    Käfigläufer-Asynchronmaschine belastet, deren Drehzahl mithilfe eines
    Frequenzumrichters gesteuert werden kann.

    Zur Bestimmung der Ersatzschaltbild-Parameter werden ein Kurzschluss- und
    ein Leerlaufversuch durchgeführt. Der Kurzschlussversuch dient der
    Bestimmung der Parameter im Längszweig, des komplexen Anlaufstromes sowie
    des Anzugmomentes. Mithilfe des Leerlaufversuches lassen sich die Parameter
    im Querzweig und der komplexe Leerlaufstrom bestimmen. Dazu wird die
    Ständerspannung verändert.

    Im nächsten Versuchsteil wird die Belastung der ASM und die synchrone
    Drehzahl untersucht. Die Messungen werden im Betriebsbereich um die
    synchrone Drehzahl aufgenommen, damit die Belastung der Maschine im
    motorischen und generatorischen Betrieb gemessen werden kann.

    Im letzten Versuchsteil wird die im Schleifringläufer induzierte Spannung
    in Abhängigkeit zur Drehzahl gemessen.

\end{abstract}

\section{Anlaufmoment und Kippmoment}

Bei diesem Versuch wurde das Drehmoment mit einem 40cm langen Stab und einer
Digitalen Wage gemessen.

Mit der Gleichung \ref{eq:moment} wurde das Drehmoment aus den Messungen berechnet.

\begin{equation} \label{eq:moment}
    M=G\cdot g\cdot l = \frac{G_{I_1}}{1000}\si{kg}\cdot 9,81\si{m/s^2} \cdot 0,4\si{m}
\end{equation}

In den Abbildungen \ref{fig:Anlaufmoment} und \ref{fig:Kippmoment} wird das
Anlaufmoment der ASM über die Spannung graphisch dargestellt.

Die obere Hälfte der Abbildungen zeigt eine linearisierte Extrapolation des
Verlaufs der Drehmomente $M_{an}$ und $M_{\textit{Kipp}}$ über die Spannung
$U_1^2$, bis $400^2V$.

Aber in der unteren Hälfte der Abbildungen wird das Drehmoment der ASM direkt
über die Spannung $U_1$ dargestellt und mit einer quadratischen Polynomfunktion
extrapoliert.

\begin{table}[htbp]
    \centering
    \begin{tabularx}{\columnwidth}{XXXXXXX}
    \toprule
     $I_1[A]$ &  $P_1[W]$ &  $U[V]$ &  $G[g]$ &  $M_{an}[Nm]$ \\
    \midrule
            1,0 &          28 &      27,5 &        35 &        0,137340 \\
            2,0 &          92 &      55,0 &        85 &        0,333540 \\
            3,0 &         194 &      82,0 &       222 &        0,871128 \\
            4,0 &         352 &     107,0 &       426 &        1,671624 \\
            4,2 &         392 &     113,0 &       464 &        1,820736 \\
            5,0 &         470 &     134,0 &       685 &        2,687940 \\
    \bottomrule
    \end{tabularx}
    \caption{Anlaufmoment}
\end{table}


\begin{figure}[htbp]
    \centering
    \includegraphics[width=\columnwidth]{./figures/anlaufmoment.pdf}
    \caption{Anlaufmoment}
    \label{fig:Anlaufmoment}
\end{figure}

Der lineare Verlauf in Abb.\ref{fig:Anlaufmoment}, $M_{an}(U)$ [\textcolor{red}{rot}]:
\begin{gather*}
    0,1511 x - 0,07306 \\
    M_{an}(U_N=400^2V^2)=24,11\ \si{Nm}
\end{gather*}

Und die quadratische Polynomfunktion $M_{an}(U)$ [\textcolor{blue}{blau}]:
\begin{gather*}
    0,000199 x^2 -0,007889 x + 0,190977\\
    M_{an}(U_N=400V)=28,83\ \si{Nm}
\end{gather*}

\begin{table}[htbp]
    \centering
    \begin{tabularx}{\columnwidth}{XXXXXXX}
        \toprule
             $I_1[A]$ &  $P_1[W]$ &  $U[V]$ &  $G[g]$ &  $M_{an}[Nm]$ \\
        \midrule
                1,0 &          24 &        36 &        32 &        0,125568 \\
                2,0 &         102 &        73 &       218 &        0,855432 \\
                3,0 &         380 &       108 &       508 &        1,993392 \\
                4,0 &         670 &       140 &       891 &        3,496284 \\
                4,2 &         745 &       148 &      1000 &        3,924000 \\
                5,0 &        1046 &       175 &      1425 &        5,591700 \\
        \bottomrule
    \end{tabularx}
    \caption{Kippmoment}
\end{table}


\begin{figure}[htbp]
    \centering
    \includegraphics[width=\columnwidth]{./figures/kippmoment.pdf}
    \caption{Kippmoment}
    \label{fig:Kippmoment}
\end{figure}

Der lineare Verlauf in Abb.\ref{fig:Kippmoment}, $M_{\textit{Kipp}}(U)$ [\textcolor{red}{rot}]:
\begin{gather*}
    0,1863 x - 0,1431\\
    M_{Kipp}(U_N=400^2V^2)=29,67 \ \si{Nm}
\end{gather*}

Und die quadratische Polynomfunktion $M_{\textit{Kipp}}(U)$ [\textcolor{blue}{blau}]:
\begin{gather*}
    0,000198 x^2 - 0,0024495 x -0,035842\\
    M_{Kipp}(U_N=400V)=30,6\ \si{Nm}
\end{gather*}

Es gibt eine Abweichung von wenigen Prozent zwischen der linearisierten und
quadratischen Extrapolation. Durch die quadratische Extrapolation bekommen
Messfehler einen stärkeren Einfluss. Eine quadratische Extrapolation eignet
sich trotzdem um eine anschauliche Gerade zu bilden.


\subsection{Zusammenhang zwischen Drehmoment und Spannung}

Bei zunehmender Spannung $U_1$ steigt das Drehmoment M quadratisch an. Das
Drehmoment ist direkt proportional zu $U^2$ ($M \sim U^2$) und darauf ergibt
sich folgende Beziehung: \[ M = \textit{konst.} \cdot U^{2} \]


\section{Bestimmung der Ersatzschaltbildparameter}
\subsection{Parameterbestimmung aufgrund der Kurzschlussmessung}

Berechnung des gemessenen Ständerwiderstands $R_{1}$ um die Maschinenerwärmung
auf $T=75^\circ$C zu berücksichtigen:

\begin{align}
    R_{1_{75}} & = R_{1_{20}}\frac{235^\circ\si{C}+75^\circ\text{C}}{235^\circ\text{C}+20^\circ\text{C}}         \\
               & = 2,32\ \Omega\cdot \frac{235^\circ\si{C}+75^\circ\text{C}}{235^\circ\text{C}+20^\circ\text{C}} \\
    \label{eq:R1_75}
               & = 2,820\ \Omega
\end{align}

Berechnen des Läuferwiderstands mithilfe der Übersetzungsverhältnis:
\begin{align}
    R_{2_{20}}^\prime & = \ddot{u}^2\cdot R_{2_{20}} = 4,7^2 \cdot 216\si{m\Omega}   = 4,77 \Omega                \\
    R_{2_{75}}^\prime & = R_{2_{20}}^\prime \frac{235^\circ\si{C}+75^\circ\si{C}}{235^\circ\si{C}+20^\circ\si{C}} \\
                      & = 4,77\ \Omega\cdot \frac{235^\circ\si{C}+75^\circ\si{C}}{235^\circ\si{C}+20^\circ\si{C}} \\
    \label{eq:R_2^prime}
                      & = 5,8\ \Omega
\end{align}

Es gilt:
\begin{equation} \label{eq:Kurzschlusswiderstand_gemessen}
    \boxed{R_k = R_{1_{75}} + R_{2_{75}}^\prime = 2,820\ \Omega + 5,8\ \Omega = 8,62\ \Omega}
\end{equation}

Aus der Kurzschlussmessung in 4.2.1 wird der Kurzschlusswiderstand $R_k$ mit:

\begin{equation} \label{eq:Kurzschlusswiderstand}
    \boxed{R_k = \frac{U_k}{I_k}\cdot cos\varphi_k}
\end{equation}

\begin{table}[htbp]
    \begin{tabularx}{\columnwidth}{XXXXX}
        \toprule
        $I_1[A]$ & $P_1[W]$ & $U[V]$ & $G[g]$ & $M_{an}[Nm]$ \\
        \midrule
        4,2      & 392      & 113    & 464    & 1,82074      \\
        \bottomrule
    \end{tabularx}
    \caption{Kurzschlussmessung 4.2.1}
    \label{tab:Kurzschlussmessung}
\end{table}

Mit den Werten aus der Tabelle \ref{tab:Kurzschlussmessung} kann $cos\varphi_k$
berechnet werden:

\begin{equation}
    \boxed{cos\varphi_k = \frac{P_1}{\sqrt{3} \cdot I_1 \cdot U}}
\end{equation}

\begin{equation} \label{eq:cosphi_solved}
    cos\varphi_k = \frac{392 \si{W}}{\sqrt{3} \cdot 4,2 \si{A} \cdot 113 \si{V}} = 0,4769
\end{equation}

Da die Maschine im Stern geschaltet ist, ergibt sich für die Spannung $U_k$:

\begin{equation} \label{eq:Uk_solved}
    \boxed{U_k = \frac{U}{\sqrt{3}}} = \frac{113 \si{V}}{\sqrt{3}} = 65,24 \si{V}
\end{equation}

In der Formel \ref{eq:Kurzschlusswiderstand} einsetzen:

\begin{equation} \label{eq:Kurzschlusswiderstand_calc}
    \boxed{R_k = \frac{65,24\si{V}}{4,2\si{A}}\cdot cos(61,51^\circ) = \underline{7,409 \Omega}}
\end{equation}

Die Streureaktanzen $X_{1\sigma}$ und $X_{2\sigma}^\prime$ werden mit:

\begin{equation} \label{eq:Streureaktanzen}
    X_{1\sigma} + X_{2\sigma}^\prime = \frac{U_k}{I_k}\cdot sin\varphi_k
\end{equation}

Da $X_{1\sigma} = X_{2\sigma}^\prime$ ist, ergibt sich:

\begin{align} \label{eq:Streureaktanzen_solved}
    \Aboxed{X_{1\sigma} & = X_{2\sigma}^\prime = \frac{U_k}{I_k}\cdot\frac{1}{2}\cdot sin\varphi_k} \\
                        & = \frac{65,24\si{V}}{3,69\si{A}}\cdot\frac{1}{2}\cdot sin(61,51^\circ)    \\
                        & =6,826\ \Omega
\end{align}

\subsubsection{Vergleich zwischen ermitteltem und gemessen Kurzschlusswiderstand}

Die Werte des berechnete Kurzschlusswiderstands
(\ref{eq:Kurzschlusswiderstand_gemessen}) haben eine Abweichung von ca. 15 \%
zu dem gemessenen Widerstand (\ref{eq:Kurzschlusswiderstand_calc}). Diese
Differenz ist zu einem großen Teil auf die gewählte Referenztemperatur von 75°C
zurück zuführen. Bei einer zügig durchgeführten Messung ist die
Wärmeentwicklung der Maschine geringer. Nimmt man also einen Temperaturanstieg
auf nur 55°C kommt man dem gemessenen Wert deutlich näher.

\begin{equation}
    R_{2_{55}}^\prime = R_{2_{20}}^\prime \cdot \dfrac{55^\circ \si{C} + 235^\circ \si{C}}{20^\circ \si{C} + 235^\circ \si{C}} = 5,43 \Omega
\end{equation}

\begin{equation}
    R_{1_{55}}^\prime = R_{1_{20}}^\prime \cdot \dfrac{55^\circ \si{C} + 235^\circ \si{C}}{20^\circ \si{C} + 235^\circ \si{C}} = 2,64 \Omega
\end{equation}

aus (\ref{eq:Kurzschlusswiderstand_gemessen}):
\begin{equation}
    \boxed{R_{k} = R_{1_{60}} + R_{2_{55}}^\prime = 2,64 \Omega + 5,43 \Omega = 8,07 \Omega}
\end{equation}


\subsection{Parameterbestimmung aufgrund der Leerlaufmessung}
\subsubsection{Maschinenparameter $X_h$ und $R_{Fe+Reib}$ aufgrund der Leerlaufmessung}

Bei dem folgenden Versuch ist Läuferseitig ein Leerlauf und somit kein
Läuferstrom $I_{2}$ zu messen. Der fließende Strom besteht ausschließlich aus
dem Ständerstrom $I_{1}$ bzw. dem Leerlaufstrom $I_{0}$.

Der verwendete Hebelarm aus den letzten Messungen wird entfernt, der Läufer
wird mit kurzen Leitungen kurzgeschlossen.  Die verwendeten Werte werden
Tabelle \ref{tab:trennung_eisen_reib} entnommen.

Die gemessene Spannung wird auf die Strangspannung umgerechnet.
Anschließende wird über eine Masche der Spannungsabfall $U_{h, Fe}$
berechnet.

\begin{equation}
    U_{Strang} = \dfrac{U}{ \sqrt{3} } = 230,94\si{V}
\end{equation}

\begin{align} \label{eq:UhFe}
    U_{h, Fe} & = U_{Strang} - U_{1} = U_{Strang} - R_{1} \cdot I_{0}        \\
              & = 230,94\si{V} - 2,32 \Omega \cdot 2,75\si{A} = 224,56\si{V}
\end{align}


\begin{equation}
    \varphi_{0} = arcos\left( \dfrac{P_{0} - P_{Cu1} }{3 \cdot U_{Strang} \cdot I_{0} }\right) = 84,35^\circ
\end{equation}


\begin{align*}
    I_{Fe+Reib} & = I_{0} \cdot cos( \varphi ) = 0,27\si{A}   \\
    I_{m}       & = I_{0} \cdot sin( \varphi ) = 2,7367\si{A}
    %    Z_{h, Fe} = \dfrac{ U_{h, Fe} }{ I_{0} } = ( \frac{1}{X_{h}} + \frac{1}{R_{Fe}} )^{-1} = 81,66 \Omega
\end{align*}

\begin{equation}
    R_{Fe+Reib} = \dfrac{ U_{h, Fe} }{ I_{Fe+Reib} } = 831,7 \Omega
    %    \boxed{ R = Z_{h, Fe} \cdot ( \frac{1}{cos( \varphi) } +1 ) = 729,606 \Omega }
\end{equation}

\begin{equation}
    X_{h} = \dfrac{ U_{h, Fe} }{ I_{m} } = 82,06 \Omega
    %\boxed{ X_{h} = 91,95 \Omega }
\end{equation}

\subsubsection{Verhältnis aus Haupt- und Streureaktanz}

mit $X_{ \sigma } =  X_{1, \sigma } + X_{2, \sigma }^\prime$

\begin{equation}
    \dfrac{ X_{h}}{X_{ \sigma }} = \dfrac{ 82,06 \Omega }{ 13,652 \Omega } = 6,01
    %\dfrac{ X_{h}}{X_{ \sigma }} = \dfrac{ 91,95 \Omega }{ 13,652 \Omega } = 6,735
\end{equation}

Die Hauptreaktanz ist ca. 6 mal größer als die Streureaktanz.

\subsubsection{Vollständige Ersatzschaltbild [ESB]}

Das ESB beschreibt immer einen Strang der Maschine, so dass die Verschaltung
(Stern- oder Dreieck) bezüglich der Messgrößen Strom und Spannung zu
berücksichtigen ist. $U_1$ und $I_1$ stellen die Stranggrößen des
Ersatzschaltbildes dar.

\begin{figure}[htbp]
    \centering
    \includegraphics[width=\columnwidth]{./figures/Vollstaendiges_ESB.jpg}
    \caption{Vollständige Ersatzschaltbild}
    \label{fig:ESB_Vollstaendig}
\end{figure}

\section{Trennung von Eisen- und Reibungsverlusten}

Die gemessenen Leerlaufleistungen $P_{0}$ beinhalten $P_{Cu}$, $P_{Reib}$ und
$P_{Fe}$. Um die Eisen- und Reibungsverluste zu trennen, werden als Erstes die
Kupferverluste$ P_{Cu}$ abgezogen.

\begin{equation}
    P_{0} - P_{Cu} = P_{Reib+Fe}
    \label{eq:P_reib_fe}
\end{equation}

\begin{table}[htbp]
    \centering
    \begin{tabularx}{\columnwidth}{XXXXXX}
        \toprule
        $I_0[A]$ &  $P_0[W]$ &  $P_{Cu}[W]$ &  $P_{Fe+Reib}$ &  $U[V]$ &  $n[min]$ \\
        & & & $[W]$& & \\
        \midrule
               2,75 &         240 &        19,1400 &            220,8600 &       400 &        1493 \\
               1,70 &         140 &        11,8320 &            128,1680 &       300 &        1490 \\
               1,10 &          80 &         7,6560 &             72,3440 &       200 &        1485 \\
               0,58 &          44 &         4,0368 &             39,9632 &       100 &        1461 \\
        \bottomrule
    \end{tabularx}
    \caption{Trennung von Eisen und Reibungsverlusten}
    \label{tab:trennung_eisen_reib}
\end{table}


Zur graphischen Ermittlung von $P_{Reib}$ werden die Messwerte für
$P_{Reib+Fe}$ nun graphisch als Funktion der quadratischen Spannung $U^2$
dargestellt.

\begin{figure}[htbp]
    \centering
    \includegraphics[width=\columnwidth]{./figures/trennung_eisen_reib.pdf}
    \caption{Trennung von Eisen- und Reibungsverlusten}
    \label{fig:trennung_von_eisen_reib}
\end{figure}

Die Messerwerte bilden eine Gerade, deren Ordinatenabschnitt $P_{Reib}$
darstellt. $P_{Reib} = 24,38 \si{W}$

Zur Berechnung von $R_{Reib}$ wird die Spannung $U_{h, Fe}$ aus {\ref{eq:UhFe}}
verwendet

\begin{equation}
    R_{Reib} = \dfrac{3\cdot U_{h, Fe}^{2}}{ P_{Reib} } = 6205 \Omega
\end{equation}

Zur Bestimmung von $R_{Fe}$ wird $P_{Fe}$ an der Stelle $U_{1N}^{2}$ abgelesen.
$P_{Fe} = 193,07 \si{W}$

\begin{equation}
    R_{Fe} = \dfrac{3\cdot U_{h,Fe}^{2}}{ P_{Fe} } = 783,56 \Omega
\end{equation}


\subsubsection{Parallelschaltung von $R_{Reib}$ und $R_{Fe}$ }

\begin{equation}
    R_{Reib}||R_{Fe} = \dfrac{ R_{Reib} \cdot R_{Fe}}{R_{Reib} + R_{Fe}} = 695,7 \Omega
\end{equation}


\subsubsection{Vergleich der Parallelschaltung von $R_{Reib}$ und $R_{Fe}$ mit
$R_{Reib+Fe}$ aus 5.3.2}

Die starke Abweichung von 136 $ \Omega$ zwischen dem über Strom und Spannung
berechneten Widerstand $R_{h, Fe}$ und dem über die Leistung berechneten
Widerstand lässt sich zum Teil mit dem sehr ungenauen analogen Wattmeter und
dem dadurch entstehenden Parallaxenfehler begründen. Durch die Quadrierung der
gemessenen Spannung verstärkt sich zusätzlich der Fehlereinfluss.

\section{$M/n$ Kennlinie}

aus (\ref{eq:Streureaktanzen_solved}), (\ref{eq:R_2^prime}) und (\ref{eq:R1_75})

\smallskip
Mit $R_1 = 2,82\Omega$:
\begin{equation} \label{eq:s_k-mit-R_1}
    s_k = \frac{R_2^{\prime}}{\sqrt{R_{1}^{2} + X_{\sigma}^{2}}} = \frac{5,80\Omega}{\sqrt{2,82^2+13,652^2}\Omega} = 0,416
\end{equation}

\begin{align}
    M_{k} & = \frac{U_{1}^{2}}{4 \pi n_{1} \left(R_{1} + \sqrt{R_{1}^{2} + X_{\sigma}^{2}}\right)}           \\
          & = \frac{400\si{V}^{2}}{4 \pi \cdot 25 \si{sec^{-1}} \cdot (2,82 + \sqrt{2,82^2+13,652^2})\Omega} \\
          & = 31,653 \si{Nm}
    \label{eq:M_k-mit-R_1}
\end{align}

\smallskip
Mit $R_1 = 0\Omega$:
\begin{equation} \label{eq:s_k-naehrung}
    s_k \approx \frac{R_2^{\prime}}{X_{\sigma}} = \frac{5,80\Omega}{13,652\Omega} = 0,425
\end{equation}

\begin{align}
    M_{k} & \approx \frac{U_{1}^{2}}{4 \pi X_{\sigma} n_{1}} \approx \frac{400\si{V}^{2}}{4 \pi\cdot 13,652 \Omega \cdot 25 \si{sec^{-1}}} \\
          & \approx 37,306 \si{Nm}
    \label{eq:M_k-mit-R_1}
\end{align}

\begin{table}[htbp]
    \centering
    \begin{tabularx}{\columnwidth}{XXX}
        \toprule
        $s$   & $M[Nm]$ & $n_1[min^{-1}]$ \\
        \midrule
        -0,50 & -29,58  & 2250            \\
        -0,25 & -30,09  & 1875            \\
        0,00  & 0       & 1500            \\
        0,25  & 30,09   & 1125            \\
        0,50  & 29,58   & 750             \\
        0,75  & 24,01   & 375             \\
        1,00  & 19,49   & 0               \\
        \bottomrule
    \end{tabularx}
    \caption{Drehmoment über die Drehzahl}
    \label{tab:sigma_vs_M_n1}
\end{table}

\begin{table}[htbp]
    \centering
    \begin{tabularx}{\columnwidth}{XXXXXXX}
        \toprule
        $s$   & $M_{kipp}[Nm]$ & $n_1[min^{-1}]$ \\
        \midrule
        -0,50 & -25,32  & 2250            \\
        -0,25 & -14,90  & 1875            \\
        0,00  & 0       & 1500            \\
        0,25  & 14,90   & 1125            \\
        0,50  & 25,32   & 750             \\
        0,75  & 30,39   & 375             \\
        1,00  & 31,65   & 0               \\
        \bottomrule
    \end{tabularx}
    \caption{Kippmoment über die Drehzahl}
    \label{tab:M_kipp-mit-laeufervorwiderstand}
\end{table}


Der eben berechnete Wert hat eine Abweichung von wenigen Prozent, dies lässt
sich mit den typischen Messungenauigkeiten begründen. Klar zu sehen ist aber
das eine Vernachlässigung von $R_{1}$ zu einem deutlichen Fehler führt.

\begin{figure}[htbp]
    \centering
    \includegraphics[width=\columnwidth]{./figures/m_n-kennlinie.pdf}
    \caption{$M/n$ Kennlinie}
    \label{fig:m_n-kennlinie}
\end{figure}


\section{Wirkleistungsfluss}

\begin{figure}[htbp]
    \centering
    \includegraphics[width=\columnwidth]{./figures/Wirkleistungsfluss.jpg}
    \caption{Wirkleistungsfluss für Synchrone Drehzahl und Leerlauf}
\end{figure}

Im Leerlauf und bei synchroner Drehzahl ist der Schlupf $s=0$. Im Läufer wird
keine Spannung induziert, das heißt es fließt kein Strom im Läufer. Somit
treten keine $P_{Cu2}$ -Verluste auf.

\begin{equation} \label{eq:Kupfer2_verluste}
    P_{Cu2} = P_{L} \cdot s
\end{equation}

Die Synchrone Drehzahl kann nur erreicht werden, wenn die ASM mit einer
Lastmaschine belastet wird, die die Leistung $P_{Reib}$ aufbringt, sodass der
Läufer die gleichen Drehzahl wie der Ständer hat.

Bei der Leerlaufmessung beinhaltet die gemessene Leistung $P_{1}$ daher
$P_{Cu1}$, $P_{Fe}$ und $P_{Reib}$. Bei der Messung bei Synchroner Drehzahl
besteht $P_{1}$ lediglich aus $P_{Cu1}$ und $P_{Fe}$.

\begin{table}[htbp]
   \centering
   \begin{tabularx}{\columnwidth}{XXXXXX}
      \toprule
      $I_1[A]$ & $P_1[W]$ & $n[min^{-1}]$ & $s$   & $P_L[W]$ & $P_{mech}[W]$ \\
      \midrule
      5,4      & 2860     & 1330                       & 0,11  & 2420,24  & 2121,11       \\
      5,0      & 2620     & 1347                       & 0,10  & 2215,43  & 1964,63       \\
      4,2      & 2040     & 1388                       & 0,07  & 1697,70  & 1546,10       \\
      3,1      & 980      & 1450                       & 0,03  & 705,63   & 657,28        \\
      2,8      & 200      & 1500                       & 0,00  & -57,05   & -81,88        \\
      3,2      & -700     & 1550                       & -0,03 & -979,70  & -1037,19      \\
      4,2      & -1570    & 1595                       & -0,06 & -1912,30 & -2058,25      \\
      5,0      & -2040    & 1625                       & -0,08 & -2444,57 & -2673,11      \\
      5,4      & -2320    & 1644                       & -0,10 & -2759,76 & -3049,53      \\
      \bottomrule
   \end{tabularx}
   \caption{$P_L$ und $P_{mech}$}
   \label{tab:drehmomente_vergleichen}
\end{table}


\begin{table}[htbp]
    \centering
    \begin{tabularx}{\columnwidth}{XXX}
        \toprule
        $n[min^{-1}]$  & $M_L[Nm]$  & $M_{mech}[Nm]$ \\
        \midrule
        1330 & 17,38  & 15,23    \\
        1347 & 15,71  & 13,93    \\
        1388 & 11,68  & 10,64    \\
        1450 & 4,65   & 4,33     \\
        1500 & -0,36  & -0,52    \\
        1550 & -6,04  & -6,39    \\
        1595 & -11,45 & -12,32   \\
        1625 & -14,37 & -15,71   \\
        1644 & -16,03 & -17,71   \\
        \bottomrule
    \end{tabularx}
    \caption{$M_L$ und $M_{mech}$}
    \label{tab:drehmomente_vergleichen}
\end{table}



\begin{figure}[htbp]
    \centering
    \includegraphics[width=\columnwidth]{./figures/moment_ueber-n_vergleichen.pdf}
    \caption{$(M_L)\text{bzw.}(M_{mech})/n$ Kennlinie}
    \label{fig:M_L-mit-M_mech-vergleichen}
\end{figure}

\section{Läuferspanung und -frequenz}

Über die Änderung der Drehzahl $n$ wird die Läuferfrequenz und damit die im
Läufer induzierte Spannung $U_2$ beeinflusst.

\begin{equation}
    f_2 = \dfrac{n_1 - n}{n_1}  \cdot f_1
\end{equation}

\begin{table}[htbp]
  \centering
  \begin{tabularx}{\columnwidth}{XXX}
    \toprule
    $n[min^{-1}]$ & $U_2[V]$ & $f_2[Hz]$ \\
    \midrule
    0             & 83,3     & 50,00     \\
    250           & 69,5     & 41,67     \\
    500           & 56,0     & 33,33     \\
    750           & 41,6     & 25,00     \\
    1000          & 27,7     & 16,67     \\
    1250          & 14,0     & 8,33      \\
    1500          & 2,1      & 0,00      \\
    1750          & 13,8     & -8,33     \\
    2000          & 28,3     & -16,67    \\
    \bottomrule
  \end{tabularx}
  \caption{Läuferspannung abhängig von der Drehzahl}
\end{table}


\begin{figure}[htbp]
    \centering
    \includegraphics[width=\columnwidth]{./figures/U2_ueber-f2_vergleichen.pdf}
    \caption{Spannung $U_2$ über $f_2$}
    \label{fig:U2_ueber-f2_vergleichen}
\end{figure}

Es ist zu sehen das die induzierte Spannung mit steigender Drehzahl also
sinkender Frequenz abnimmt, dieser Zusammenhang geht aus dem induktionsgesetz
hervor wonach die induzierte Spannung abhängig von der relativ Bewegung ist.

Die Läuferfrequenz sinkt bis ins negative. Diese negative Frequenz gibt den
Generatorbetrieb an, weswegen man hier auch wieder eine steigende
Läuferspannung messen kann.

\vspace{5mm}
\subsubsection{ Vergleich des errechneten und gemessenen
Übersetzungsverhältnisses $ \ddot{u} $}

Will man aus den gemessenen Werten das Übersetzungsverhältnis ü berechnen
benötigt man die Spannung $U_{20}$ für eine Drehzahl von $ n = 0 $
beziehungsweise der synchronfrequenz von $ f_{1} = 50Hz $. Die Ständerspannung
$ U_{1} $ liegt an 400V an.

Vergleicht man das mit den Werten des Datenblatts errechnete
Übersetzungsverhältnis von 4,77 mit dem gemessenen von 4,8 liegt ein Fehler
von knapp 1\%  vor.

\begin{equation}
    \ddot{u} = \frac{ U_{1} }{ U_{20} } = \frac{ 400V }{ 83,3V } = 4,80
\end{equation}

\newpage
\section{Weitere Fragen zur ASM}
\subsection{Möglichkeiten zur Drehzahländerung einer ASM}

\begin{equation}
    n = \frac{f_{1}}{p} \cdot (1-s)
\end{equation}

Diese Gleichung zeigt drei Möglichkeiten zur Änderung der Drehzahl auf.

\begin{enumerate}
    \item Änderung der Speisefrequenz mittels Frequenzumrichter
    \item Änderung der Polpaarzahl
          \begin{equation}
              n \sim \frac{1}{p}
          \end{equation}
          Die Drehzahl ist indirekt proportional zur Polpaarzahl.
    \item Änderung des Schlupfs
          Die Änderung der Statorspannung bei einer gleichbleibenden Belastung
          führt zu einer Änderung des Schlupfs. Um im stabilen Betriebsbereich
          zu bleiben, muss $s<s_{k}$ sein.
\end{enumerate}

\subsection{Warum ist der Leerlaufstrom $I_{0}$ der ASM nicht Null? Welchen
    Hauptzweck erfüllt er?}

Der Leerlaufstrom $I_{0}$ dient der Magnetisierung der Hauptreaktanz und kann
daher als Magnetisierungsstrom $I_{M}$ interpretiert werden. Im Leerlauf kann
das ESB auf den Ständerwicklungswiderstand und die Hauptreaktanz
zusammengefasst werden. Daher nimmt die Maschine im Leerlauf nahezu
ausschließlich Blindleistung auf.

\subsection{Welche Frequenz hat der Strom im Läufer einer ASM im Stillstand?}

Im Stillstand ist der Schlupf $s=1$. Der Läufer steht still und die
Läuferströme haben Netzfrequenz. Die ASM verhält sich im Stillstand wie ein
sekundärseitig kurzgeschlossener Transformator. Der Betriebspunkt der ASM im
Stillstand wird als Kurzschlusspunkt der ASM bezeichnet.

\subsection{Welches Drehmoment liefert eine ASM, wenn sie mit synchroner
    Drehzahl läuft?}

Bei synchroner Drehzahl wird keine Spannung im Läufer induziert, weshalb kein
Strom im Läufer fließt. Das Drehmoment ist abhängig vom Stromfluss: $M \sim
    I^{2}$

Daher geht das Drehmoment bei synchroner Drehzahl gegen Null.




\end{document}
