\documentclass[conference]{IEEEtran}
\IEEEoverridecommandlockouts
% The preceding line is only needed to identify funding in the first footnote. If that is unneeded, please comment it out.
\usepackage[ngerman]{babel}
\usepackage{cite}
\usepackage{amsmath,amssymb,amsfonts,mathtools}
\usepackage{algorithmic}
\usepackage{graphicx}
\usepackage{textcomp}
\usepackage[dvipsnames]{xcolor}
\def\BibTeX{{\rm B\kern-.05em{\sc i\kern-.025em b}\kern-.08em
    T\kern-.1667em\lower.7ex\hbox{E}\kern-.125emX}}
\usepackage{listings}
\usepackage{tabularx}
\usepackage{booktabs}
\usepackage{siunitx}
\usepackage{float}
\usepackage{caption}

\definecolor{codegreen}{rgb}{0,0.6,0}
\definecolor{codegray}{rgb}{0.5,0.5,0.5}
\definecolor{codepurple}{rgb}{0.58,0,0.82}
\definecolor{backcolour}{rgb}{0.95,0.95,0.95}
\renewcommand{\lstlistingname}{Code}% Listing -> Algorithm
\renewcommand{\lstlistlistingname}{List of \lstlistingname s}% List of Listings -> List of Algorithms
\lstdefinestyle{mystyle}{
    backgroundcolor=\color{backcolour},
    commentstyle=\color{codegreen},
    keywordstyle=\color{magenta},
    numberstyle=\tiny\color{codegray},
    stringstyle=\color{codepurple},
    basicstyle=\ttfamily\footnotesize,
    breakatwhitespace=false,
    breaklines=true,
    captionpos=b,
    keepspaces=true,
    numbers=left,
    numbersep=3pt,
    showspaces=false,
    showstringspaces=false,
    showtabs=false,
    tabsize=3
}
\lstset{style=mystyle}

\usepackage{hyperref}
\hypersetup{
    pdftex,
    pdftitle={Versuch GM-Klein},
    pdfsubject={Versuch GM-Klein},
    pdfauthor=AJC,
    colorlinks,
    citecolor=black,
    filecolor=black,
    linkcolor=black,
    urlcolor=black
}

\author{
    \IEEEauthorblockN{
        \textsc{Ayham Alhalaibi}
    }
    \and
    \IEEEauthorblockN{
        \textsc{Julia Blechle}
    }
    \and
    \IEEEauthorblockN{
        \textsc{Clara Huber}
    }
}

\begin{document}

\title{
    \centering
    \includegraphics[width=0.5\textwidth]{../OTHR_OTHR_Logo.pdf}\\
    \textsc{DC machine - small} \\
}

\maketitle

\begin{abstract}

\end{abstract}

\section{Idle meassuring}
\subsection{electric motor force}
The machine is operated without (a) load. This way the Voltage $U_q$ which is induced in the generator machine can be meassured.
The induced voltage is also called electric motor force. For this experiment the initial 5000rpm will be reduced by 1000rpm each time.

At 5000rpm the motor has its maximum voltage $U_A = 11.94V$. The machine constant consists of an experimentaly determined Konstant c (diffrend for each machine) and the magnetic flux $\Phi$ $ c \cdot \Phi_E$.\\

\begin{equation}
    c \cdot \Phi_E = \frac{U_q}{n} = \frac{8,8V}{4000/60s} = 0,132Vs
\end{equation}

\section{loaded generator mode}
The generator is now operated at 4000rpm. 





\end{document}
